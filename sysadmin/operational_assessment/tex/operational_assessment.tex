\documentclass{article}   	% use "amsart" instead of "article" for AMSLaTeX format
\usepackage[margin=0.5in]{geometry}                		% See geometry.pdf to learn the layout options. There are lots.
\geometry{a4paper}                   		% ... or a4paper or a5paper or ...

\usepackage[parfill]{parskip}    		% Activate to begin paragraphs with an empty line rather than an indent
\usepackage{graphicx}				% Use pdf, png, jpg, or eps with pdflatex; use eps in DVI mode
\usepackage{enumerate}								% TeX will automatically convert eps --> pdf in pdflatex		


\title{Operational Review\\ IN719 Systems Administration}
\date{}							% Activate to display a given date or no date

\begin{document}
\maketitle


On Monday, 22nd May at noon we will begin our approximately two week operations evaluation period which will end on Friday, 2nd June at 17:00.  During this time we will not meet as a class.

You will be evaluated on the following criteria:

\vspace{0.5cm}

1.  \textbf{Infrastructure Uptime and ownCloud Setup (50 marks):}  Your primary goal is to keep your infrastructure running, which includes all components set up prior to this assessment. In addition I will give you a task that involves setting up a new solution, ownCloud, with a given (feasible) deadline.  If you keep your infrastructure running for 267 (99\%) hours during this period (including ownCloud from the setup deadline onwards), you will receive 50 marks.  If you have less than 217 hours of uptime (80\%) you will receive 0 marks.  Every hour between the two is worth 1 mark.  Uptime is measured by an alternative monitoring system that does not interfere with your configuration.  Note that you will not lose marks if there is an outage in vCloud or the Polytechnic network -- but if you observe any error you should verify whether this is actually the case.

If you do the arithmetic you'll see that there are two hours unaccounted for.  We will get to that.

\vspace{0.5cm}

2.  \textbf{Issues/Tickets (20 marks):}  During the evaluation period you will receive tickets to handle.  Some tickets will come from a standard user (user@op-bit.nz).  You are only expected to deal with those tickets during normal business hours (9:00-17:00 Monday-Friday).  You will also receive tickets from your manager (it-manager@op-bit.nz).  You are expected to handle those tickets between 8:00 and 22:00 \underline{any day} of the week. (Note that this is independent from your system uptime. Your systems and services must be running 24 hours a day!) 

For these tickets you will be evaluated based on:

\begin{itemize}
\item effectively resolving the issues;
\item handling them promptly;
\item communicating with relevant parties.
\end{itemize}

\vspace{0.5cm}

3.  \textbf{Security Breach/System Failure (20 marks):}  At some point during the evaluation period you will have to deal with a major systems failure or a security breach.  (This is where the two hours of downtime come in).  You will be evaluated on:

\begin{itemize}
\item recognising the problem;
\item solving the problem;
\item communicating with affected parties (e.g., the two email addresses above).
\end{itemize}

\vspace{0.5cm}

4.  \textbf{Bacula/Nagios/Puppet setup (10 marks):} I will review your utilisation of Bacula, Nagios, and Puppet.  Note that I will do this near the end of the evaluation period, so you should use the time to improve your use of these tools if necessary.

During the evaluation (and before it starts) I will be available to answer questions but not actively assist you in troubleshooting.

You should post a roster on your team wiki identifying a primary contact person from your team for every hour of the assessment. It is up to you how to manage this.
\end{document}
