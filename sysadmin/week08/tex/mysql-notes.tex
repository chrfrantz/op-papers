% Beamer slide template prepared by Tom Clark <tom.clark@op.ac.nz>
% Otago Polytechnic
% Dec 2012

\documentclass[10pt]{beamer}
%\usetheme{Dunedin}
\usepackage{graphicx}
\usepackage{fancyvrb}
\newcommand\codeHighlight[1]{\textcolor[rgb]{1,0,0}{\textbf{#1}}}

\title{MySQL Monitoring in Nagios}
%\subtitle{So Your Stuff Can look Nifty}
\author[IN719]{Systems Administration}
\institute[Otago Polytechnic]{
  School of Information Technology \\
  Otago Polytechnic \\
  Dunedin, New Zealand \\
}
\date{}
\begin{document}

%----------- titlepage ----------------------------------------------%
\begin{frame}[plain]
  \titlepage
\end{frame}


%----------- slide --------------------------------------------------%
\begin{frame}[fragile]
  \frametitle{in /etc/nagios-plugins/config/mysql.cfg}

\begin{Verbatim}[commandchars=\\\[\]]
# 'check_mysql_cmdlinecred' command definition
define command{
  command_name  check_mysql_cmdlinecred
  command_line  /usr/lib/nagios/plugins/check_mysql
                -H '$HOSTADDRESS$' -u '$ARG1$' -p '$ARG2$'
}



\end{Verbatim}
\end{frame}
%----------- slide --------------------------------------------------%
\begin{frame}[fragile]
  \frametitle{Use this new check}

Redefine the MySQL service:
\begin{Verbatim}[commandchars=\\\[\]]
#check that mysql services are running
define service {
	...
    check_command    \codeHighlight[check_mysql_cmdlinecred!$USER3$!$USER4$]
}

\end{Verbatim}
\end{frame}

%----------- slide --------------------------------------------------%
\begin{frame}[fragile]
  \frametitle{in /etc/nagios3/resources.cfg}

This file contains items we need, but that need to be handled carefully, like
usernames and passwords.  Add:

\begin{Verbatim}[commandchars=\\\[\]]
  # Store some usernames and passwords (hidden from the CGIs)
  # MySQL username and password
  \codeHighlight[$USER3$]=nagioscheck
  \codeHighlight[$USER4$]=foo

}
\end{Verbatim}

Why don't we need to hard-code the db server's name or IP?

\end{frame}


\end{document}
