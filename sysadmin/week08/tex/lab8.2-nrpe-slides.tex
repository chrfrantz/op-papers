% Beamer slide template prepared by Tom Clark <tom.clark@op.ac.nz>
% Otago Polytechnic
% Dec 2012

\documentclass[10pt]{beamer}
%\usetheme{Dunedin}
\usepackage{graphicx}
\usepackage{fancyvrb}
\newcommand\codeHighlight[1]{\textcolor[rgb]{1,0,0}{\textbf{#1}}}

\title{Nagios Checks with NRPE}
%\subtitle{So Your Stuff Can look Nifty}
\author[IN719]{Systems Administration}
\institute[Otago Polytechnic]{
  School of Information Technology \\
  Otago Polytechnic \\
  Dunedin, New Zealand \\
}
\date{}
\begin{document}

%----------- titlepage ----------------------------------------------%
\begin{frame}[plain]
  \titlepage
\end{frame}

%----------- slide --------------------------------------------------%
\begin{frame}
  \frametitle{Three kinds of Nagios Checks}


\begin{enumerate}
  \item Local services
  \item Network exposed services
  \item Remote services
\end{enumerate}

\end{frame}


%----------- slide --------------------------------------------------%
\begin{frame}
  \frametitle{Remote Services}


\begin{itemize}
  \item Sometimes we want to monitor remote services that are not exposed on a network
  \item There are a few ways to handle this, each with its pros and cons
  \item We'll consider one way, using \emph{NRPE}
\end{itemize}

\end{frame}

%----------- slide --------------------------------------------------%
\begin{frame}
  \frametitle{NRPE}

The approach
\begin{itemize}
  \item We install the NRPE daemon on the remote host we want to monitor
  \item We also install the desired monitoring plugins on the remote host
  \item On the Nagios server, we use the NRPE plugin to send the check request
        to the remote host
  \item On the remote host, the NRPE daemon runs the check and reports the results
        back to the Nagios server
\end{itemize}
\end{frame}

%----------- slide --------------------------------------------------%
\begin{frame}
  \frametitle{Install the NRPE daemon}

On the remote server you want to monitor
\begin{itemize}
	\item install \texttt{nagios-nrpe-server}
  \item Edit \texttt{/etc/nagios/nrpe.cfg}
  \item Add \texttt{mgmt} to the \texttt{allowed\_hosts} values
  \item Add the following command to the commands section \\
   \texttt{command[check\_disk]=/usr/lib/nagios/plugins/check\_disk -w 20\% -c 10\%}

  \item Restart the NRPE server
\end{itemize}
\end{frame}

%----------- slide --------------------------------------------------%
\begin{frame}
  \frametitle{On the Nagios server}

  \begin{itemize}
	  \item Define a service that checks remote disks over NRPE.
	  \item Define a new hostgroup.
	  \item Put the remote servers on which we want to monitor disk
		  space in the host group.
  \end{itemize}
\end{frame}
%----------- slide --------------------------------------------------%
\begin{frame}
  \frametitle{Monitoring Windows Hosts}

This done with a tool that is similar to NRPE
\begin{itemize}
  \item On the Windows host, we install NSClient++
  \item On the Nagios server, we use the \texttt{check\_nt} plugin to communicate with
        the Windows host
\end{itemize}
\end{frame}

\end{document}
