% Beamer slide template prepared by Tom Clark <tom.clark@op.ac.nz>
% Otago Polytechnic
% Dec 2012

\documentclass[10pt]{beamer}
\usetheme{CambridgeUS}
\usepackage{graphicx}
\usepackage{fancyvrb}
\usepackage{ulem}
\newcommand\codeHighlight[1]{\textcolor[rgb]{1,0,0}{\textbf{#1}}}

\title{Backups - The Most Important Topic of the Semester}
\author[IN719]{Systems Administration}
\institute[Otago Polytechnic]{
  Otago Polytechnic \\
  Dunedin, New Zealand \\
}
\date{}

\begin{document}

%----------- titlepage ----------------------------------------------%
\begin{frame}[plain]
  \titlepage
\end{frame}

%----------- slide --------------------------------------------------%
\begin{frame}
  \frametitle{The Sysadmin's Lifeboat}
 
Good backups solve a range of problems.

\begin{itemize}
\item Somebody deletes important files:  restore from backup.
\item Your server's hard drive burns up:  restore from backup.
\item Bad guys break into your server:  wipe it and restore from backup.
\item Your office burns down: set up in a new location, restore from backup.
\item You need to archive old data: store it on backup media
\end{itemize}
\end{frame}

%----------- slide --------------------------------------------------%
\begin{frame}
  \frametitle{Start with a plan}
 

\begin{itemize}
\item What to back up
\item When to back up
\item How to store and manage your backup media
\item How to restore from backup
\end{itemize}

\end{frame}

%----------- slide --------------------------------------------------%
\begin{frame}
  \frametitle{Backup everything!}
 

\begin{itemize}
\item Whatever you may need, you'll have it.
\item It's easy to restore after a major problem. \\
       However,
\item You'll wind up storing a lot of data, much of it redundant.
\item Larger backups are more time consuming to create and to restore.
\item You may not want to retain some data.
\end{itemize}
\end{frame}


%----------- slide --------------------------------------------------%
\begin{frame}
  \frametitle{Backup selectively}
 

\begin{itemize}
\item Identify data that is valuable.
\item Identify data that you can't get elsewhere.
\item Identify data that you don't want to save.
\item Consult applicable regulations and policies.
\end{itemize}
\end{frame}

%----------- slide --------------------------------------------------%
\begin{frame}
  \frametitle{When to back up}
 

\begin{itemize}
\item  Should you back up hourly, daily, weekly?
\item  Do you want full or incremental backups?
\item  How much can you afford to lose?
\item  How long does it take to perform a backup?
\item  Do you need to take any services offline while backing up? 
\end{itemize}
\end{frame}


%----------- slide --------------------------------------------------%
\begin{frame}
  \frametitle{Managing backup media}
 

\begin{itemize}
\item We may back up to tape, disk, or an offsite service. 
\item We need a rotation schedule for media.
\item We must maintain a catalogue of backup media.
\end{itemize}
\end{frame}

%----------- slide --------------------------------------------------%
\begin{frame}
\frametitle{Backup and Rotation Strategies}

There are many different strategies. Currently popular:

\begin{itemize}
\item Home/Small Office (Disk Storage): 3-2-1 

$\rightarrow$ 3 total backups, 2 on disk (1 online, 1 offline), 1 off site
\item Enterprise (traditionally for Tape Storage): Grandfather-Father-Son 

$\rightarrow$ 3 tape sets that are changed on daily, weekly, and monthly basis (+ quarterly, yearly cold storage); specification of tape retention rules
\end{itemize}

A more comprehensive overview: \url{http://searchdatabackup.techtarget.com/tip/An-introduction-to-data-backup-tape-rotation-schemes}

\end{frame}

%----------- slide --------------------------------------------------%
\begin{frame}
  \frametitle{Restoring from backup}
 

\begin{itemize}
\item There's no point in backing up if we can't restore.
\item Typically a complicated process.
      \begin{itemize}
        \item Identify storage media
	\item Mount and restore in the correct order
	\item Execute restore commands
      \end{itemize}
\item The process needs to be well documented and tested regularly.
\end{itemize}
\end{frame}


%----------- slide --------------------------------------------------%
\begin{frame}
  \frametitle{A backup system is needed}

\begin{itemize}
\item It identifies what is to be backed up.
\item It performs the backups automatically according to a schedule.
\item It manages storage media.
\item It helps automate the restore process.
\end{itemize}
\end{frame}


%----------- slide --------------------------------------------------%
\begin{frame}
  \frametitle{Many systems are available}

\begin{itemize}
\item ARCServe
\item Arkeia
\item Barracuda
\item AMANDA
\item Bacula
\end{itemize}
\end{frame}

%----------- slide --------------------------------------------------%
\begin{frame}
  \frametitle{Bacula}

  We will use Bacula, which is documented here:
  
  http://www.bacula.org/5.0.x-manuals/en/main/main/index.html

\end{frame}

\end{document}
