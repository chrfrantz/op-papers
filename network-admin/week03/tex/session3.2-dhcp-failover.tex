% Beamer slide template prepared by Tom Clark <tom.clark@op.ac.nz>
% Otago Polytechnic
% Dec 2012

\documentclass[10pt]{beamer}
\usetheme{Dunedin}
\usepackage{graphicx}
\usepackage{fancyvrb}
\usepackage{hyperref}

\newcommand\codeHighlight[1]{\textcolor[rgb]{1,0,0}{\textbf{#1}}}

\title{DHCP Failover}

\author[IN715]{Networks Administration}
\institute[Otago Polytechnic]{
  Otago Polytechnic \\
  Dunedin, New Zealand \\
}
\date{}
\begin{document}

%----------- titlepage ----------------------------------------------%
\begin{frame}[plain]
  \titlepage
\end{frame}

\begin{frame}
	\frametitle{Need for failover}
	\begin{itemize}
	    \item It should be pretty clear that DHCP servers need to just work all the time.
	    \item In general, they do.  If you're using something like the ISC DHCP server on BSD, there's virtually no chance that it's just going to go down without warning.
	    \item For this reason, there hasn't been much failover capability in DHCP servers.
	    \item However, there are still plenty of things that can break even if your software stack is very solid.
	    \end{itemize}
\end{frame}

\begin{frame}
	\frametitle{The problem}
	\begin{itemize}
		\item The problem is that DHCP servers hold a lot of \emph{state} about the network.
		\item In particular, they keep track of current leases\footnote{In \texttt{/var/db/dhcp.leases}}.
			\item If we just run two DHCP servers side-by-side, their lease databases will conflict.
			
	\end{itemize}
\end{frame}

\begin{frame}
	\frametitle{The solution}
	\begin{itemize}
		\item If we're going to run two DHCP servers in a failover setup, they need to know about each other.
		\item They need to sync their databases.
		\item This capability was added to \texttt{dhcpd} in version 3\footnote{Little problem: You are running version 2.}
	\end{itemize}
\end{frame}

\begin{frame}[fragile]
	\frametitle{Primary server configuration}
	\begin{verbatim}
	failover peer "dhcp-failover" {
	    primary; # declare ourselves primary
	    address 172.16.5.10;
	    port 520;
	    peer address 172.16.5.2;
	    peer port 520;
	    max-response-delay 10;
	    max-unacked-updates 10;
	    load balance max seconds 3;
	    mclt 1800;
	    split 128;
	}
	\end{verbatim}
\end{frame}

\begin{frame}[fragile]
	\frametitle{Secondary server configuration}
	\begin{verbatim}
	failover peer "dhcp-failover" {
	    secondary; # declare ourselves secondary
	    address 172.16.5.2;
	    port 520;
  	    peer address 172.16.5.10;
	    peer port 520;
	    max-response-delay 10;
	    max-unacked-updates 10;
	    load balance max seconds 3;
	}
	\end{verbatim}

\end{frame}

\begin{frame}[fragile]
	\frametitle{Addition to subnet configuration}
	\begin{verbatim}
	subnet 172.16.5.0 netmask 255.255.255.0 {
	    pool{
	        failover peer "dhcp-failover";
	        range 172.16.5.50 172.16.5.100;
	    }
	}
	\end{verbatim}

\end{frame}
\begin{frame}
	Questions?
\end{frame}
\end{document}
